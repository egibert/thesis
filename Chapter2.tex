\section{Introduction}
%In this chapter, we present our approach to providing a programmable and dynamic framework that can support data-driven applications by extending software-defined environment concepts to drive the process of dynamically composing infrastructure services from multiple providers. The resulting distributed software-defined environment (dSDE) autonomously evolves over the application life cycle while meeting objectives and constraints set by users, applications, and/or resource providers. In Section 3.1.1, we first discuss the requirements for a dSDE followed by a summary of our methodology in Section 3.1.2. We then present the architecture and implementation of the three layers necessary to realize a dSDE in Sections 3.2-3.4. In Section 3.5, we then demonstrate how the three layers work together to provide a dSDE using a rule-engine–based and a constraint-programming–based approaches. Finally, we provide a summary of the chapter in Section 3.6.

\section{Framework}

\subsection{Location-aware Overlay Network Layer}

\subsection{Content-based Routing Layer}\label{sec:frameworkc}

\subsection{Serverless Messaging Layer}\label{sec:serverless}

\subsection{Memory-mapped Streaming Analytics Pipeline}

\subsubsection{Data Collection Layer}

\subsubsection{Data Processing Layer}

\subsubsection{Data Storage and Query Layer}

\subsection{Rule-based Programming Abstraction}\label{sec:programming-data}

\subsubsection{Content-Driven Rule-Based System}

\subsubsection{Data Quality Rule-Based System}

%\section{Semantics of the Programming Abstraction}
%\section{Realizing Data-Drive Edge Stream Processing}
%\subsection{Basic Functionality}
%\section{Experimental Evaluation}
%\subsection{Scalability and Overhead Experiments}
%\subsection{Performance experiments}
%\section{Summary}