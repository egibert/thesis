\section{Challenges of IoT Applications}
IoT data features several all the Vs of BigData:
\begin{itemize}
    \item Volume: 
    \item Velocity:
    \item Variety: 
    \item Veracity: 
\end{itemize}

In addition IoT introduces a new set of challenges that big data does not have. The main challenges associated with the development and deployment of IoT analytics applications are:

\begin{itemize}
    \item Data heterogeneity:
    \item Real-time nature:
    \item Time and location dependencies:
    \item Privacy and security sensitivity:
\end{itemize}

%\section{IoT Application Lifecycle}
%The IoT application lifecycle comprises of the following phases: 

%\begin{itemize}
%    \item Data Collection:
%    \item Data Analysis:
%    \item Data Storage:
%\end{itemize}

\section{Motivation}

Due to the proliferation of the Internet of Things (IoT) paradigm, the number of devices connected to the Internet is growing. These devices are generating unprecedented amounts of data at the edges of the infrastructure. Although the generated data provides great potential, identifying and processing relevant data points hidden in streams of unimportant data, and doing this in near real time, remains a significant challenge. Existing stream processing platforms require the data to be transported to the cloud for processing, resulting in latencies that can prevent timely decision making or may reduce the amount of data processed.



\section{Problem Description}
\section{Overview of Thesis Research}
\section{Contributions}
This dissertation makes the following contributions:
\begin{itemize}
  \item A content- and location-based programming abstraction for specifying \textbf{what} data gets collected and \textbf{where} the data gets analyzed.
  \item A rule-based programming abstraction for specifying \textbf{when} to trigger data-processing tasks based on data observations.
  \item A programming abstraction for specifying \textbf{how} to split a given dataflow and place operators across edge and cloud resources.
  \item An operator placement strategy that aims to minimize an aggregate cost which covers the end-to-end latency (time for an event to traverse the entire dataflow), the data transfer rate (amount of data transferred between the edge and the cloud) and the messaging cost (number of messages transferred between edge and the cloud).
  \item Performance optimizations on the data-processing pipeline in order to achieve real-time performance on constrained devices.
  \item An implementation of the above capabilities as part of the R-Pulsar software stack and its evaluation using embedded devices (Raspberry Pi and Android phone).
\end{itemize}

\section{Outline}
The rest of this thesis is organized as follows.

Chapter 2 shows the benefits of using data staging, as well as motivating and representative workflow examples. This chapter also summarizes the key parameters and components of coupled scientific workflows.

Chapter 3 presents an overview of related research work.

Chapter x summarizes the research work of this thesis and presents future research
directions.