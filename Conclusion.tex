\section{Perspective}
\section{Conclusion}
\section{Future Work}
In this section, we discuss the current challenges and limitations of the research in the edge computing area, and we provide future directions and potential starting points for those challenges. 
\\\\
\textbf{Energy Management:} Powering billions of embedded devices deployed into the environment is one of the biggest challenges that IoT faces. IoT devices are limited in power, so deploying these devices for a very long time becomes the main hurdle. It is estimated that up to 80\% of the total energy consumption of an embedded system is due to software-related activities~\cite{5402965}. None of the surveyed systems have the ability to quantify the amount of energy spend or to schedule computations while being energy efficient. Energy management needs to be incorporated in all the layers in the edge middleware architecture, since there is a need for tools that will give feedback on how energy efficient the code is. For those reasons energy management is a potential research direction. 
\\\\
\textbf{Real-Time:} Unlike the well-provisioned cloud servers, edge devices do not support heavyweight software due to limited computational capabilities. For instance, a typical edge device used for performing edge analytics is a Raspberry Pi 3, which is not sufficient for executing complex data processing tools such as Apache Spark (requires at least 8-core CPU and 8GB of memory for good performance). Most of the middleware surveyed in this paper relay on software that was designed to be run on large clusters, instead of building on software that was designed for constrained devices, not allowing to achieve real-time. Another research direction is the need for new lightweight libraries and algorithms specifically designed for edge devices, instead of reusing software that was designed to be deployed in large clusters.
\\\\
\textbf{Scalability:} The number of IoT devices are increasing day by day all over the world, which raises a new issue of scalability. Thus, we need to deal with the scaling of devices and services in the edge computing environment. Most of the middleware surveyed in this paper relies on the cloud to achieve scalability, limiting the amount of IoT devices that will be able to support. Scalability also needs to be integrated into all of the layers since the number of devices in IoT systems is often dramatically larger than in traditional computing environments, potentially reaching hundreds, thousands, or even millions. Another potential research direction is the need for more distributed architectures such as P2P to be able to support large number of IoT devices at once. 
\\\\
\textbf{Mobility:} Unlike a traditional sensor network which is usually operated by a single organization, IoT sensors are owned and controlled by different individual users. Hence, IoT introduces mobility and makes it totally uncontrollable and hard to predict. Very few systems surveyed offer the ability to monitor the sensors location and move the computations closer to it. Mobility needs to be incorporated in the programming model in order to move computations in order to achieve the low latency requirements that IoT applications need. A research direction is the development of live migration techniques that will forecast the movement in the near future of operators, reducing the down time of the migration.
\\\\
\textbf{Security:} The remote management of complex installations of IoT devices in environments such as factories, power plants, or oil rigs requires major attention to security. Remote actuation and programming capabilities can pose high-security risks. Cryptographic protocols for transport layer security, security certificates, physical isolation, and other established industry practices play a critical role in this area, but various interesting technical challenges remain. A possible research direction is to create algorithms that provide strong security protection, while fitting within an acceptable footprint. It needs to be lightweight enough that will still leave room for the embedded OS and applications code. 