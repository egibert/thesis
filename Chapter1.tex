\section{AR} 
Associative Rendezvous (AR) is a paradigm for content-based decoupled interactions with programmable reactive behaviors. Rendezvous-based interactions~\cite{AR} provide a mechanism for decoupling senders and receivers. The Rendezvous Point (RP) is the node where rendezvous interactions occur, and it can be an edge node or a cloud node. Senders transmit messages to an RP without knowledge of who or where the receivers are. The associative rendezvous interaction model consists of three elements, messages, associative selection, and reactive behaviors, which are described below. The AR interaction model consists of three elements: messages, associative selection, and reactive behaviors, which are described below~\cite{meteor2008}.

\subsection{AR Message}
An AR message~\cite{meteor2008} is defined as the triplet: (header, action, data). The data field may be empty or may contain the message payload. The header includes a semantic profile in addition to the credentials of the sender, a message context, and the Lifetime of the message. The profile is a set of attributes and/or attribute–value pairs, and defines the set of recipients of the message. The attributes are keywords from the application information space, such as the type of data a sensor produces (temperature or humidity) and/or its location, the type of functionality a service provides and/or its quality of service guarantees, and the capability and/or the cost of a resource, while the values field may be a keyword, partial keyword, wildcard, or range from the same space. At the RP, a profile is classified as a data profile or an interest profile depending on the action field of the message. The data profile corresponds to a message carrying data to be stored in the system. The interest profile corresponds to a query.

\subsection{Associative Selection} 

Profiles are represented using a hierarchical schema that can be efficiently stored and evaluated by the selection engine at runtime. Profile $p$ represents a path in the hierarchical schema, [$e_0$ $\bigtriangleup$ ... $e_k$], where $e_i$ is an element operand and $\bigtriangleup$ can be a parent-child ("/") operator (i.e. at adjacent levels) or an ancestor-descendant ("//") operator (i.e. separated by more than one level). Within a level, the profile defines a propositional expression where $\bigtriangleup$ represents propositional operators, such as $\wedge$ and $\vee$ between elements at the same level. Note that the propositional expression at a level must be evaluated as TRUE for the evaluation to continue to the next level. The elements of the profile can be an attribute, $e_i$ : ($a_i$), or an attribute-value pair $e_i$ : ($a_i$, $v_i$), where $a_i$ is a keyword and $v_i$ may be a keyword, partial keyword, wildcard, or range. The singleton attribute $a_i$ evaluates to true if and only if $p$ contains the simple attribute $a_i$. The attribute-value pair ($a_i$,$u_i$) evaluates to true with respect to a profile $p$, if and only if $p$ contains attribute $a_i$ and the corresponding value $v_i$ satisfies $u_i$. For example, the profile in Figure~\ref{fig:data_profile} will be matched by the profile in Figure~\ref{fig:interest_profile}, since (1) both have matched singleton attribute video camera; (2) for attribute location, 100 - 105, satisfies the range relation; (3) for attribute type, {\it Public} matches the wildcard *; and (4) quality$=$720p satisfies the request quality$>=$720p. 

\subsection{Reactive Behaviors} 
The action field of the message defines the reactive behavior of the profile when a match occurs. Every time an RP receives an AR message, the profile is stored and matched against existing profiles. If there is a match, the action of the profile is carried out. Basic reactive behaviors are store, retrieve, notify, and delete.

The notify and delete actions are explicitly invoked on a data or an interest profile. The store
action stores the data and data profile at the RP. It also causes the message profile to be matched
against existing interest profiles with notify data action, and the data to be sent to the data consumers
that requested it in case of a positive match.

The retrieve action retrieves data corresponding to each matching data profile. The notify action
matches the message profile against existing interest/data profile, and notifies the sender if there
is at least one positive match. 

The notify action comes in two flavors: notify data and notify interest.
Notify data is used by data consumers, who want to be notified when data matching their interest
profile are stored in the system. Notify interest can be used by data producers, who want to be notified
when there is interest in the data they produce, so that they can start sending data into the system.

Finally, the delete action deletes all matching interest/data profiles. Note that the actions will
only be executed if the message header contains an appropriate credential. Also note that each
message is stored at the rendezvous for a period corresponding to the Lifetime defined in its header.
In case of multiple matches, the profiles matching are processed in random order.

\subsection{Associative Rendezvous 2.0}\label{sec:semantics}
R-Pulsar uses a custom implementation of the original AR semantics~\cite{meteor2008}. In our new implementation of the AR model, we modified to elements: the AR Message and the Reactive Behaviors.

In the AR message is now defined as a quintuplet instead of the original triplet: (header, action, data, location, and data-processing task). The location and data processing task fields have been added in to the already existing fields.

The location field has been added so the AR message can be routed based on the location and the content, where the original version of AR only routes messages based on the content. The location coordinates represent the physical location of the sensors or the physical location of where to deploy data-processing tasks and the tag helps decide where to deploy data-processing tasks, either the edge or the cloud, allowing to pick from multiple cloud or edge geographically distributed resources. 

In the new Associative Rendezvous, the actions are classified in two two different classes: resource actions and function actions. 

Resource actions are designated for discovering, starting, and stopping sensors from transmitting data. Basic resource reactive behaviors currently defined include {\it notify\_interest, notify\_data, query\_data}, and {\it delete}. The {\it notify\_interest} is used by sensors for advertising its data producing capabilities and that they want to be notified when there is someone interested in the data they can produce. The {\it notify\_data} are used by the data-processing tasks that will consume the data produced by sensors. When a {\it notify\_data} profile and a {\it notify\_interest} profile match, Figure~\ref{fig:samples_profiles} sensors are notified to start streaming data to the consumer. The {\it query\_data } action is for performing SQL-like queries on stored data. The {\it delete} action deletes all matching profiles from the system.

Function actions are designated for storing, triggering, and stopping data-processing tasks. Basic function reactive behaviors currently defined include {\it store\_function, start\_function and stop\_function}. The {\it store\_function} action allows users to submit and store user-defined data-processing tasks in the RPs, allowing to share and discover existing data-processing tasks previously uploaded by other users. This avoids the need to rewrite the same function multiple times and facilitates the reproducibility of the experiments. The {\it start\_function} allows users to trigger data-processing tasks on demand. %It also causes the function profile to be matched against existing function profiles. 
If there is a match between two profiles, the data-processing task is executed. The {\it stop\_function} allows users to stop data-processing tasks that are running. %When a store\_function profile and a start\_function profile match, the function stored in the RP is triggered. 

\begin{figure}
    \centering
    \begin{subfigure}[t]{0.5\textwidth}
        \centering
        \includegraphics[height=1in]{Figures/data_profile.pdf}
        \caption{}
        \label{fig:data_profile}
    \end{subfigure}%
    ~ 
    \begin{subfigure}[t]{0.5\textwidth}
        \centering
        \includegraphics[height=1in]{Figures/interest_profile.pdf}
        \caption{}
        \label{fig:interest_profile}
    \end{subfigure}
    \caption{Sample profiles: (a) producer profile from a video camera sensor; (b) consumer profile interested in video camera sensors}
\end{figure}


\begin{figure}
\centering
\begin{subfigure}[b]{0.9\textwidth}
   \includegraphics[width=1\linewidth]{Figures/single.pdf}
   \label{fig:Ng1} 
   \caption{}
\end{subfigure}
\begin{subfigure}[b]{0.9\textwidth}
   \includegraphics[width=1\linewidth]{Figures/multi.pdf}
   \label{fig:Ng2}
   \caption{}
\end{subfigure}
\caption{R-Pulsar space filling curve routing using simple (a) and complex (b) profiles.}
\end{figure}
